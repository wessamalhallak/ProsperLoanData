\documentclass[]{article}
\usepackage{lmodern}
\usepackage{amssymb,amsmath}
\usepackage{ifxetex,ifluatex}
\usepackage{fixltx2e} % provides \textsubscript
\ifnum 0\ifxetex 1\fi\ifluatex 1\fi=0 % if pdftex
  \usepackage[T1]{fontenc}
  \usepackage[utf8]{inputenc}
\else % if luatex or xelatex
  \ifxetex
    \usepackage{mathspec}
  \else
    \usepackage{fontspec}
  \fi
  \defaultfontfeatures{Ligatures=TeX,Scale=MatchLowercase}
\fi
% use upquote if available, for straight quotes in verbatim environments
\IfFileExists{upquote.sty}{\usepackage{upquote}}{}
% use microtype if available
\IfFileExists{microtype.sty}{%
\usepackage{microtype}
\UseMicrotypeSet[protrusion]{basicmath} % disable protrusion for tt fonts
}{}
\usepackage[margin=1in]{geometry}
\usepackage{hyperref}
\hypersetup{unicode=true,
            pdftitle={Exploring Loan Data},
            pdfauthor={Wessam Alhallak},
            pdfborder={0 0 0},
            breaklinks=true}
\urlstyle{same}  % don't use monospace font for urls
\usepackage{color}
\usepackage{fancyvrb}
\newcommand{\VerbBar}{|}
\newcommand{\VERB}{\Verb[commandchars=\\\{\}]}
\DefineVerbatimEnvironment{Highlighting}{Verbatim}{commandchars=\\\{\}}
% Add ',fontsize=\small' for more characters per line
\usepackage{framed}
\definecolor{shadecolor}{RGB}{248,248,248}
\newenvironment{Shaded}{\begin{snugshade}}{\end{snugshade}}
\newcommand{\AlertTok}[1]{\textcolor[rgb]{0.94,0.16,0.16}{#1}}
\newcommand{\AnnotationTok}[1]{\textcolor[rgb]{0.56,0.35,0.01}{\textbf{\textit{#1}}}}
\newcommand{\AttributeTok}[1]{\textcolor[rgb]{0.77,0.63,0.00}{#1}}
\newcommand{\BaseNTok}[1]{\textcolor[rgb]{0.00,0.00,0.81}{#1}}
\newcommand{\BuiltInTok}[1]{#1}
\newcommand{\CharTok}[1]{\textcolor[rgb]{0.31,0.60,0.02}{#1}}
\newcommand{\CommentTok}[1]{\textcolor[rgb]{0.56,0.35,0.01}{\textit{#1}}}
\newcommand{\CommentVarTok}[1]{\textcolor[rgb]{0.56,0.35,0.01}{\textbf{\textit{#1}}}}
\newcommand{\ConstantTok}[1]{\textcolor[rgb]{0.00,0.00,0.00}{#1}}
\newcommand{\ControlFlowTok}[1]{\textcolor[rgb]{0.13,0.29,0.53}{\textbf{#1}}}
\newcommand{\DataTypeTok}[1]{\textcolor[rgb]{0.13,0.29,0.53}{#1}}
\newcommand{\DecValTok}[1]{\textcolor[rgb]{0.00,0.00,0.81}{#1}}
\newcommand{\DocumentationTok}[1]{\textcolor[rgb]{0.56,0.35,0.01}{\textbf{\textit{#1}}}}
\newcommand{\ErrorTok}[1]{\textcolor[rgb]{0.64,0.00,0.00}{\textbf{#1}}}
\newcommand{\ExtensionTok}[1]{#1}
\newcommand{\FloatTok}[1]{\textcolor[rgb]{0.00,0.00,0.81}{#1}}
\newcommand{\FunctionTok}[1]{\textcolor[rgb]{0.00,0.00,0.00}{#1}}
\newcommand{\ImportTok}[1]{#1}
\newcommand{\InformationTok}[1]{\textcolor[rgb]{0.56,0.35,0.01}{\textbf{\textit{#1}}}}
\newcommand{\KeywordTok}[1]{\textcolor[rgb]{0.13,0.29,0.53}{\textbf{#1}}}
\newcommand{\NormalTok}[1]{#1}
\newcommand{\OperatorTok}[1]{\textcolor[rgb]{0.81,0.36,0.00}{\textbf{#1}}}
\newcommand{\OtherTok}[1]{\textcolor[rgb]{0.56,0.35,0.01}{#1}}
\newcommand{\PreprocessorTok}[1]{\textcolor[rgb]{0.56,0.35,0.01}{\textit{#1}}}
\newcommand{\RegionMarkerTok}[1]{#1}
\newcommand{\SpecialCharTok}[1]{\textcolor[rgb]{0.00,0.00,0.00}{#1}}
\newcommand{\SpecialStringTok}[1]{\textcolor[rgb]{0.31,0.60,0.02}{#1}}
\newcommand{\StringTok}[1]{\textcolor[rgb]{0.31,0.60,0.02}{#1}}
\newcommand{\VariableTok}[1]{\textcolor[rgb]{0.00,0.00,0.00}{#1}}
\newcommand{\VerbatimStringTok}[1]{\textcolor[rgb]{0.31,0.60,0.02}{#1}}
\newcommand{\WarningTok}[1]{\textcolor[rgb]{0.56,0.35,0.01}{\textbf{\textit{#1}}}}
\usepackage{graphicx,grffile}
\makeatletter
\def\maxwidth{\ifdim\Gin@nat@width>\linewidth\linewidth\else\Gin@nat@width\fi}
\def\maxheight{\ifdim\Gin@nat@height>\textheight\textheight\else\Gin@nat@height\fi}
\makeatother
% Scale images if necessary, so that they will not overflow the page
% margins by default, and it is still possible to overwrite the defaults
% using explicit options in \includegraphics[width, height, ...]{}
\setkeys{Gin}{width=\maxwidth,height=\maxheight,keepaspectratio}
\IfFileExists{parskip.sty}{%
\usepackage{parskip}
}{% else
\setlength{\parindent}{0pt}
\setlength{\parskip}{6pt plus 2pt minus 1pt}
}
\setlength{\emergencystretch}{3em}  % prevent overfull lines
\providecommand{\tightlist}{%
  \setlength{\itemsep}{0pt}\setlength{\parskip}{0pt}}
\setcounter{secnumdepth}{0}
% Redefines (sub)paragraphs to behave more like sections
\ifx\paragraph\undefined\else
\let\oldparagraph\paragraph
\renewcommand{\paragraph}[1]{\oldparagraph{#1}\mbox{}}
\fi
\ifx\subparagraph\undefined\else
\let\oldsubparagraph\subparagraph
\renewcommand{\subparagraph}[1]{\oldsubparagraph{#1}\mbox{}}
\fi

%%% Use protect on footnotes to avoid problems with footnotes in titles
\let\rmarkdownfootnote\footnote%
\def\footnote{\protect\rmarkdownfootnote}

%%% Change title format to be more compact
\usepackage{titling}

% Create subtitle command for use in maketitle
\newcommand{\subtitle}[1]{
  \posttitle{
    \begin{center}\large#1\end{center}
    }
}

\setlength{\droptitle}{-2em}

  \title{Exploring Loan Data}
    \pretitle{\vspace{\droptitle}\centering\huge}
  \posttitle{\par}
    \author{Wessam Alhallak}
    \preauthor{\centering\large\emph}
  \postauthor{\par}
      \predate{\centering\large\emph}
  \postdate{\par}
    \date{11 März 2019}


\begin{document}
\maketitle

\hypertarget{exploring-loan-data-by-wessam-alhallak}{%
\section{Exploring Loan Data by Wessam
Alhallak}\label{exploring-loan-data-by-wessam-alhallak}}

This data set contains 113,937 loans with 81 variables on each
loan,including loan amount,borrower rate (orinterest rate), currentloan
status, borrowerincome, borroweremployment status,borrower credit
history,and the latest paymentinformation. And I will list my interested
Variables after reading the descriptions of each varaibles

\begin{Shaded}
\begin{Highlighting}[]
\NormalTok{intData=RawData}\OperatorTok\KeywordTok{select}\NormalTok{(ListingCreationDate,}
\NormalTok{                         Term,}
\NormalTok{                         LoanStatus,}
\NormalTok{                         IncomeRange,}
\NormalTok{                         BorrowerRate,}
\NormalTok{                         ProsperRating..Alpha.,}
\NormalTok{                         Occupation,}
\NormalTok{                         EmploymentStatus,}
\NormalTok{                         EmploymentStatusDuration,}
\NormalTok{                         DebtToIncomeRatio,}
\NormalTok{                         IncomeRange,}
\NormalTok{                         LoanOriginalAmount,}
\NormalTok{                         LP_InterestandFees,}
\NormalTok{                         Investors)}
\end{Highlighting}
\end{Shaded}

\#Scatter matrix and first overview Here we will make an overview for
all variable together then we will try to explore variables as
univariable to know some information about the chartatsitic for each
variable , later on we will try to explore some information about the
realtion between each two variables and then we try to get in depth
between more than two variables in mulitvariable section finally I will
reflect my conclusinios in the last section \#conclusion

\begin{Shaded}
\begin{Highlighting}[]
\KeywordTok{ggpairs}\NormalTok{(intData[}\KeywordTok{sample.int}\NormalTok{(}\KeywordTok{nrow}\NormalTok{(intData),}\DecValTok{1000}\NormalTok{),],}\DataTypeTok{cardinality_threshold=}\DecValTok{100}\NormalTok{)}
\end{Highlighting}
\end{Shaded}

\includegraphics{prosperLoanData_files/figure-latex/unnamed-chunk-2-1.pdf}

\hypertarget{univariate-plot}{%
\section{Univariate Plot}\label{univariate-plot}}

\begin{Shaded}
\begin{Highlighting}[]
\KeywordTok{ggplot}\NormalTok{(}\DataTypeTok{data=}\NormalTok{RawData,}\KeywordTok{aes}\NormalTok{(}\DataTypeTok{x=}\NormalTok{ListingCreationDate))}\OperatorTok{+}
\StringTok{  }\KeywordTok{geom_histogram}\NormalTok{(}\DataTypeTok{bins=}\DecValTok{200}\NormalTok{,}\DataTypeTok{fill=}\StringTok{"#567456"}\NormalTok{)}\OperatorTok{+}
\StringTok{  }\KeywordTok{theme}\NormalTok{(}\DataTypeTok{axis.text.x =} \KeywordTok{element_text}\NormalTok{(}\DataTypeTok{size=}\DecValTok{10}\NormalTok{),}
          \DataTypeTok{axis.text.y =} \KeywordTok{element_text}\NormalTok{(}\DataTypeTok{size=}\DecValTok{10}\NormalTok{, }\DataTypeTok{angle=}\DecValTok{45}\NormalTok{))}\OperatorTok{+}
\StringTok{  }\KeywordTok{labs}\NormalTok{(}\DataTypeTok{title=}\StringTok{"Listing Creation Date  "}\NormalTok{,}
       \DataTypeTok{subtitle=}\StringTok{"The date the listing was created."}\NormalTok{,}
       \DataTypeTok{y=}\StringTok{"Count"}\NormalTok{)}
\end{Highlighting}
\end{Shaded}

\includegraphics{prosperLoanData_files/figure-latex/unnamed-chunk-3-1.pdf}

\begin{Shaded}
\begin{Highlighting}[]
\NormalTok{Occupation_group<-}\StringTok{ }
\StringTok{  }\KeywordTok{group_by}\NormalTok{(RawData,Occupation)}

\NormalTok{RawData.Occupation_group<-}
\StringTok{  }\KeywordTok{summarise}\NormalTok{(Occupation_group,}\DataTypeTok{count=}\KeywordTok{n}\NormalTok{())}

\NormalTok{RawData.Occupation_group<-}
\StringTok{  }\KeywordTok{arrange}\NormalTok{(RawData.Occupation_group,}\OperatorTok{-}\NormalTok{(count))}



\KeywordTok{ggplot}\NormalTok{(}\DataTypeTok{data =} \KeywordTok{head}\NormalTok{(RawData.Occupation_group,}\DecValTok{10}\NormalTok{),}
       \KeywordTok{aes}\NormalTok{(}\DataTypeTok{x=} \KeywordTok{reorder}\NormalTok{(Occupation, }\OperatorTok{-}\NormalTok{count), }
           \DataTypeTok{y=}\NormalTok{count))}\OperatorTok{+}
\StringTok{  }\KeywordTok{geom_bar}\NormalTok{(}\DataTypeTok{stat =} \StringTok{"identity"}\NormalTok{)}\OperatorTok{+}
\StringTok{  }\KeywordTok{theme}\NormalTok{(}\DataTypeTok{axis.text.x =} \KeywordTok{element_text}\NormalTok{(}\DataTypeTok{size=}\DecValTok{10}\NormalTok{, }\DataTypeTok{angle=}\DecValTok{45}\NormalTok{,}\DataTypeTok{hjust =} \DecValTok{1}\NormalTok{))}\OperatorTok{+}
\StringTok{  }\KeywordTok{labs}\NormalTok{(}\DataTypeTok{title=}\StringTok{"Occupation "}\NormalTok{,}
       \DataTypeTok{subtitle=}\StringTok{"The Occupation selected by the Borrower at the time they created the listing."}\NormalTok{,}
       \DataTypeTok{x=}\StringTok{"Occupation"}\NormalTok{,}
       \DataTypeTok{y=}\StringTok{"Count"}\NormalTok{)}
\end{Highlighting}
\end{Shaded}

\includegraphics{prosperLoanData_files/figure-latex/unnamed-chunk-4-1.pdf}

\begin{Shaded}
\begin{Highlighting}[]
\NormalTok{RawData }\OperatorTok\StringTok{ }\KeywordTok{count}\NormalTok{(Occupation) }\OperatorTok\StringTok{ }
\StringTok{  }\KeywordTok{mutate}\NormalTok{(}\DataTypeTok{Occupation=}\KeywordTok{reorder}\NormalTok{(Occupation,}\OperatorTok{-}\NormalTok{n))}\OperatorTok\StringTok{ }
\StringTok{  }\KeywordTok{filter}\NormalTok{(n}\OperatorTok{>}\KeywordTok{quantile}\NormalTok{(n,}\FloatTok{0.80}\NormalTok{))}\OperatorTok
\StringTok{  }\KeywordTok{ggplot}\NormalTok{(}\KeywordTok{aes}\NormalTok{(Occupation, }\DataTypeTok{y =}\NormalTok{ n,}\DataTypeTok{fill=}\KeywordTok{sqrt}\NormalTok{(n)))}\OperatorTok{+}
\StringTok{    }\KeywordTok{geom_bar}\NormalTok{(}\DataTypeTok{stat =} \StringTok{"identity"}\NormalTok{)}\OperatorTok{+}
\StringTok{    }\KeywordTok{theme_minimal}\NormalTok{()}\OperatorTok{+}
\StringTok{    }\KeywordTok{theme}\NormalTok{(}\DataTypeTok{axis.text.x =} \KeywordTok{element_text}\NormalTok{(}\DataTypeTok{size =} \DecValTok{10}\NormalTok{, }\DataTypeTok{angle=} \DecValTok{45}\NormalTok{, }\DataTypeTok{hjust=}\DecValTok{1}\NormalTok{))}\OperatorTok{+}
\StringTok{  }\KeywordTok{labs}\NormalTok{(}\DataTypeTok{title=}\StringTok{"Occupation "}\NormalTok{, }\DataTypeTok{x=}\StringTok{"Occupation"}\NormalTok{, }\DataTypeTok{y=}\StringTok{"Count"}\NormalTok{,}\DataTypeTok{fill=}\StringTok{"Sqrt(Count)"}\NormalTok{)}
\end{Highlighting}
\end{Shaded}

\includegraphics{prosperLoanData_files/figure-latex/unnamed-chunk-7-1.pdf}

\begin{Shaded}
\begin{Highlighting}[]
\NormalTok{RawData }\OperatorTok
\StringTok{  }\CommentTok{#mutate(as.factor(Term))%>%}
\StringTok{  }\CommentTok{#count(Term) %>% }
\StringTok{  }\KeywordTok{ggplot}\NormalTok{(}\KeywordTok{aes}\NormalTok{(}\DataTypeTok{x=}\KeywordTok{as.factor}\NormalTok{(Term)))}\OperatorTok{+}
\StringTok{    }\KeywordTok{geom_bar}\NormalTok{()}\OperatorTok{+}
\StringTok{    }\KeywordTok{coord_trans}\NormalTok{(}\DataTypeTok{y=}\StringTok{'sqrt'}\NormalTok{)}\OperatorTok{+}
\StringTok{    }\KeywordTok{theme_minimal}\NormalTok{()}\OperatorTok{+}
\StringTok{    }\KeywordTok{theme}\NormalTok{(}\DataTypeTok{axis.text.x =} \KeywordTok{element_text}\NormalTok{(}\DataTypeTok{size =} \DecValTok{24}\NormalTok{))}\OperatorTok{+}
\StringTok{    }\KeywordTok{labs}\NormalTok{(}\DataTypeTok{title=}\StringTok{"Term"}\NormalTok{, }\DataTypeTok{x=}\StringTok{"Term"}\NormalTok{, }\DataTypeTok{y=}\StringTok{"Count"}\NormalTok{)}
\end{Highlighting}
\end{Shaded}

\includegraphics{prosperLoanData_files/figure-latex/unnamed-chunk-8-1.pdf}

\begin{Shaded}
\begin{Highlighting}[]
\NormalTok{loanstat=RawData}\OperatorTok{$}\NormalTok{LoanStatus}
\NormalTok{RawData }\OperatorTok
\StringTok{  }\KeywordTok{count}\NormalTok{(LoanStatus) }\OperatorTok\StringTok{ }
\StringTok{  }\KeywordTok{mutate}\NormalTok{(}\DataTypeTok{LoanStatus=}\KeywordTok{reorder}\NormalTok{(LoanStatus,}\OperatorTok{-}\NormalTok{n))}\OperatorTok\StringTok{ }
\StringTok{  }\CommentTok{#filter(n>quantile(n,0.20))%>%}
\StringTok{  }\KeywordTok{ggplot}\NormalTok{(}\KeywordTok{aes}\NormalTok{(}\DataTypeTok{x=}\NormalTok{LoanStatus, }\DataTypeTok{y =} \KeywordTok{sqrt}\NormalTok{(n),}\DataTypeTok{fill=}\KeywordTok{sqrt}\NormalTok{(n)))}\OperatorTok{+}
\StringTok{    }\KeywordTok{geom_bar}\NormalTok{(}\DataTypeTok{stat =} \StringTok{"identity"}\NormalTok{)}\OperatorTok{+}
\StringTok{    }\CommentTok{#coord_trans(x='sqrt')+}
\StringTok{    }\CommentTok{#coord_polar()+}
\StringTok{    }\KeywordTok{scale_fill_viridis}\NormalTok{(}\DataTypeTok{option =} \StringTok{"A"}\NormalTok{)}\OperatorTok{+}
\StringTok{    }\KeywordTok{theme_minimal}\NormalTok{()}\OperatorTok{+}
\StringTok{    }\KeywordTok{theme}\NormalTok{(}\DataTypeTok{axis.text.x =} \KeywordTok{element_text}\NormalTok{(}\DataTypeTok{size =}\DecValTok{8}\NormalTok{, }\DataTypeTok{angle=} \DecValTok{45}\NormalTok{, }\DataTypeTok{hjust=}\DecValTok{1}\NormalTok{))}\OperatorTok{+}
\StringTok{    }\KeywordTok{labs}\NormalTok{(}\DataTypeTok{title=}\StringTok{"Loan Status Count "}\NormalTok{, }\DataTypeTok{x=}\StringTok{"Loan Status"}\NormalTok{, }\DataTypeTok{y=}\StringTok{"Count"}\NormalTok{,}\DataTypeTok{fill=}\StringTok{"Sqrt(Count)"}\NormalTok{)}
\end{Highlighting}
\end{Shaded}

\includegraphics{prosperLoanData_files/figure-latex/unnamed-chunk-9-1.pdf}

\begin{Shaded}
\begin{Highlighting}[]
\NormalTok{IncomeRange=RawData}\OperatorTok{$}\NormalTok{IncomeRange}
\NormalTok{RawData }\OperatorTok
\StringTok{  }\KeywordTok{count}\NormalTok{(IncomeRange) }\OperatorTok\StringTok{ }
\StringTok{  }\CommentTok{#mutate(IncomeRange=reorder(IncomeRange,-n))%>% }
\StringTok{  }\CommentTok{#filter(n>quantile(n,0.20))%>%}
\StringTok{  }\KeywordTok{ggplot}\NormalTok{(}\KeywordTok{aes}\NormalTok{(}\DataTypeTok{x=}\NormalTok{IncomeRange, }\DataTypeTok{y =}\NormalTok{ n,}\DataTypeTok{fill=}\KeywordTok{sqrt}\NormalTok{(n)))}\OperatorTok{+}
\StringTok{    }\KeywordTok{geom_bar}\NormalTok{(}\DataTypeTok{stat =} \StringTok{"identity"}\NormalTok{)}\OperatorTok{+}
\StringTok{    }\CommentTok{#coord_trans(x='log')+}
\StringTok{    }\CommentTok{#coord_polar()+}
\StringTok{    }\CommentTok{#scale_fill_viridis(option = "A")+}
\StringTok{    }\KeywordTok{theme_minimal}\NormalTok{()}\OperatorTok{+}
\StringTok{    }\KeywordTok{theme}\NormalTok{(}\DataTypeTok{axis.text.x =} \KeywordTok{element_text}\NormalTok{(}\DataTypeTok{size =}\NormalTok{ , }\DataTypeTok{angle=} \DecValTok{45}\NormalTok{, }\DataTypeTok{hjust=}\DecValTok{1}\NormalTok{))}\OperatorTok{+}
\StringTok{    }\KeywordTok{labs}\NormalTok{(}\DataTypeTok{title=}\StringTok{"Income Range"}\NormalTok{,}
         \DataTypeTok{subtitle =} \StringTok{"The income range of the borrower at the time the listing was created."}\NormalTok{,}
         \DataTypeTok{x=}\StringTok{"Income Range"}\NormalTok{,}
         \DataTypeTok{y=}\StringTok{"Count"}\NormalTok{,}
         \DataTypeTok{fill=}\StringTok{"Sqrt(Count)"}\NormalTok{)}
\end{Highlighting}
\end{Shaded}

\includegraphics{prosperLoanData_files/figure-latex/unnamed-chunk-10-1.pdf}

EmploymentStatusDuration The length in months of the employment status
at the time the listing was created.

\begin{Shaded}
\begin{Highlighting}[]
\NormalTok{RawData}\OperatorTok
\StringTok{  }\KeywordTok{filter}\NormalTok{(}\KeywordTok{is.finite}\NormalTok{(EmploymentStatusDuration))}\OperatorTok
\StringTok{  }\KeywordTok{ggplot}\NormalTok{(}\KeywordTok{aes}\NormalTok{(}\DataTypeTok{x=}\NormalTok{EmploymentStatusDuration))}\OperatorTok{+}
\StringTok{  }\KeywordTok{geom_histogram}\NormalTok{(}\DataTypeTok{binwidth =} \DecValTok{10}\NormalTok{,}\DataTypeTok{fill=}\StringTok{"#563ec3"}\NormalTok{)}\OperatorTok{+}
\StringTok{   }\KeywordTok{theme_minimal}\NormalTok{()}\OperatorTok{+}
\StringTok{    }\KeywordTok{theme}\NormalTok{(}\DataTypeTok{axis.text.x =} \KeywordTok{element_text}\NormalTok{(}\DataTypeTok{size =}\NormalTok{ , }\DataTypeTok{angle=} \DecValTok{45}\NormalTok{, }\DataTypeTok{hjust=}\DecValTok{1}\NormalTok{))}\OperatorTok{+}
\StringTok{    }\KeywordTok{labs}\NormalTok{(}\DataTypeTok{title=}\StringTok{"Income Range"}\NormalTok{,}
         \DataTypeTok{subtitle =} \StringTok{"The length in months of the employment status at the time the listing was created."}\NormalTok{,}
         \DataTypeTok{x=}\StringTok{"Inocme Range"}\NormalTok{,}
         \DataTypeTok{y=}\StringTok{"Count"}\NormalTok{,}
         \DataTypeTok{fill=}\StringTok{"Sqrt(Count)"}\NormalTok{)}
\end{Highlighting}
\end{Shaded}

\includegraphics{prosperLoanData_files/figure-latex/unnamed-chunk-11-1.pdf}

BorrowerRate

The Borrower's interest rate for this loan.

\begin{Shaded}
\begin{Highlighting}[]
\NormalTok{BorrowerRate=RawData}\OperatorTok{$}\NormalTok{BorrowerRate}
\NormalTok{RawData}\OperatorTok
\StringTok{  }\KeywordTok{filter}\NormalTok{(BorrowerRate}\OperatorTok{>}\KeywordTok{quantile}\NormalTok{(BorrowerRate,}\FloatTok{0.01}\NormalTok{))}\OperatorTok
\StringTok{  }\KeywordTok{filter}\NormalTok{(BorrowerRate}\OperatorTok{<}\KeywordTok{quantile}\NormalTok{(BorrowerRate,}\FloatTok{0.99}\NormalTok{))}\OperatorTok
\StringTok{  }\CommentTok{#count(BorrowerRate)%>%}
\StringTok{  }\KeywordTok{ggplot}\NormalTok{(}\KeywordTok{aes}\NormalTok{(}\DataTypeTok{x=}\NormalTok{BorrowerRate))}\OperatorTok{+}
\StringTok{    }\KeywordTok{geom_histogram}\NormalTok{(}\DataTypeTok{bins=}\DecValTok{40}\NormalTok{,}\DataTypeTok{fill=}\StringTok{"#563ec3"}\NormalTok{)}\OperatorTok{+}
\StringTok{    }\KeywordTok{theme_minimal}\NormalTok{()}\OperatorTok{+}
\StringTok{    }\KeywordTok{theme}\NormalTok{(}\DataTypeTok{axis.text.x =} \KeywordTok{element_text}\NormalTok{(}\DataTypeTok{size =}\NormalTok{ , }\DataTypeTok{angle=} \DecValTok{45}\NormalTok{, }\DataTypeTok{hjust=}\DecValTok{1}\NormalTok{))}\OperatorTok{+}
\StringTok{    }\KeywordTok{labs}\NormalTok{(}\DataTypeTok{title=}\StringTok{"The Borrower's interest rate for this loan.  "}\NormalTok{, }\DataTypeTok{x=}\StringTok{"Borrower Rate"}\NormalTok{, }\DataTypeTok{y=}\StringTok{"Count"}\NormalTok{)}
\end{Highlighting}
\end{Shaded}

\includegraphics{prosperLoanData_files/figure-latex/unnamed-chunk-12-1.pdf}

LoanOriginalAmount

\begin{Shaded}
\begin{Highlighting}[]
\NormalTok{RawData}\OperatorTok
\StringTok{  }\KeywordTok{filter}\NormalTok{(LoanOriginalAmount}\OperatorTok{!=}\DecValTok{0}\NormalTok{)}\OperatorTok
\StringTok{  }\KeywordTok{ggplot}\NormalTok{(}\KeywordTok{aes}\NormalTok{(}\DataTypeTok{x =}\NormalTok{ LoanOriginalAmount))}\OperatorTok{+}
\StringTok{  }\KeywordTok{geom_density}\NormalTok{()}
\end{Highlighting}
\end{Shaded}

\includegraphics{prosperLoanData_files/figure-latex/unnamed-chunk-13-1.pdf}

Prosper Rating : The Prosper Rating assigned at the time the listing was
created between AA - HR. Applicable for loans originated after July
2009.

\begin{Shaded}
\begin{Highlighting}[]
\CommentTok{#levels(RawData$ProsperRating..Alpha.)}
\NormalTok{RawData }\OperatorTok
\StringTok{  }\KeywordTok{filter}\NormalTok{(RawData}\OperatorTok{$}\NormalTok{ProsperRating..Alpha.}\OperatorTok{!=}\StringTok{""}\NormalTok{)}\OperatorTok
\StringTok{  }\CommentTok{#filter(RawData$ListingCreationDate>2009)}
\StringTok{  }\KeywordTok{count}\NormalTok{(ProsperRating..Alpha.) }\OperatorTok\StringTok{ }
\StringTok{  }\CommentTok{#mutate(IncomeRange=reorder(IncomeRange,-n))%>% }
\StringTok{  }\CommentTok{#filter(n>quantile(n,0.20))%>%}
\StringTok{  }\KeywordTok{ggplot}\NormalTok{(}\KeywordTok{aes}\NormalTok{(}\DataTypeTok{x=}\NormalTok{ProsperRating..Alpha., }\DataTypeTok{y =}\NormalTok{ n,}\DataTypeTok{fill=}\KeywordTok{sqrt}\NormalTok{(n)))}\OperatorTok{+}
\StringTok{    }\KeywordTok{geom_bar}\NormalTok{(}\DataTypeTok{stat =} \StringTok{"identity"}\NormalTok{)}\OperatorTok{+}
\StringTok{    }\KeywordTok{theme_minimal}\NormalTok{()}\OperatorTok{+}
\StringTok{    }\KeywordTok{theme}\NormalTok{(}\DataTypeTok{axis.text.x =} \KeywordTok{element_text}\NormalTok{( ))}\OperatorTok{+}
\StringTok{    }\KeywordTok{labs}\NormalTok{(}\DataTypeTok{title=}\StringTok{"Prosper Rating"}\NormalTok{, }\DataTypeTok{x=}\StringTok{"Prosper Rating"}\NormalTok{, }\DataTypeTok{y=}\StringTok{"Count"}\NormalTok{,}\DataTypeTok{fill=}\StringTok{"Sqrt(Count)"}\NormalTok{)}
\end{Highlighting}
\end{Shaded}

\includegraphics{prosperLoanData_files/figure-latex/unnamed-chunk-15-1.pdf}

Monthly Loan Payment

\begin{Shaded}
\begin{Highlighting}[]
\NormalTok{MonthlyLoanPayment=RawData}\OperatorTok{$}\NormalTok{MonthlyLoanPayment}

\NormalTok{RawData}\OperatorTok
\StringTok{  }\KeywordTok{filter}\NormalTok{(MonthlyLoanPayment}\OperatorTok{>}\KeywordTok{quantile}\NormalTok{(MonthlyLoanPayment,}\FloatTok{0.01}\NormalTok{))}\OperatorTok
\StringTok{  }\KeywordTok{filter}\NormalTok{(MonthlyLoanPayment}\OperatorTok{<}\KeywordTok{quantile}\NormalTok{(MonthlyLoanPayment,}\FloatTok{0.99}\NormalTok{))}\OperatorTok
\StringTok{  }\KeywordTok{ggplot}\NormalTok{(}\KeywordTok{aes}\NormalTok{(}\DataTypeTok{x=}\NormalTok{MonthlyLoanPayment))}\OperatorTok{+}
\StringTok{    }\KeywordTok{geom_histogram}\NormalTok{(}\DataTypeTok{bins=}\DecValTok{100}\NormalTok{,}\DataTypeTok{fill=}\StringTok{"#563ec3"}\NormalTok{)}\OperatorTok{+}
\StringTok{    }\KeywordTok{theme_minimal}\NormalTok{()}\OperatorTok{+}
\StringTok{    }\KeywordTok{theme}\NormalTok{(}\DataTypeTok{axis.text.x =} \KeywordTok{element_text}\NormalTok{(}\DataTypeTok{size =}\NormalTok{ , }\DataTypeTok{angle=} \DecValTok{45}\NormalTok{, }\DataTypeTok{hjust=}\DecValTok{1}\NormalTok{))}\OperatorTok{+}
\StringTok{    }\KeywordTok{labs}\NormalTok{(}\DataTypeTok{title=}\StringTok{"The scheduled monthly loan payment."}\NormalTok{, }\DataTypeTok{x=}\StringTok{"Monthly Loan Payment"}\NormalTok{, }\DataTypeTok{y=}\StringTok{"Count"}\NormalTok{)}
\end{Highlighting}
\end{Shaded}

\includegraphics{prosperLoanData_files/figure-latex/unnamed-chunk-16-1.pdf}

\begin{Shaded}
\begin{Highlighting}[]
\NormalTok{RawData}\OperatorTok
\StringTok{  }\KeywordTok{filter}\NormalTok{(MonthlyLoanPayment}\OperatorTok{>}\DecValTok{125}\NormalTok{)}\OperatorTok
\StringTok{  }\KeywordTok{filter}\NormalTok{(MonthlyLoanPayment}\OperatorTok{<}\DecValTok{200}\NormalTok{)}\OperatorTok
\StringTok{  }\KeywordTok{ggplot}\NormalTok{(}\KeywordTok{aes}\NormalTok{(}\DataTypeTok{x=}\NormalTok{MonthlyLoanPayment))}\OperatorTok{+}
\StringTok{    }\KeywordTok{geom_histogram}\NormalTok{(}\DataTypeTok{bins=}\DecValTok{300}\NormalTok{,}\DataTypeTok{fill=}\StringTok{"#563ec3"}\NormalTok{)}\OperatorTok{+}
\StringTok{    }\KeywordTok{theme_minimal}\NormalTok{()}\OperatorTok{+}
\StringTok{    }\KeywordTok{theme}\NormalTok{(}\DataTypeTok{axis.text.x =} \KeywordTok{element_text}\NormalTok{(}\DataTypeTok{size =}\NormalTok{ , }\DataTypeTok{angle=} \DecValTok{45}\NormalTok{, }\DataTypeTok{hjust=}\DecValTok{1}\NormalTok{))}\OperatorTok{+}
\StringTok{    }\KeywordTok{labs}\NormalTok{(}\DataTypeTok{title=}\StringTok{"The scheduled monthly loan payment"}\NormalTok{, }\DataTypeTok{x=}\StringTok{"Monthly Loan Payment"}\NormalTok{, }\DataTypeTok{y=}\StringTok{"Count"}\NormalTok{)}
\end{Highlighting}
\end{Shaded}

\includegraphics{prosperLoanData_files/figure-latex/unnamed-chunk-17-1.pdf}

CreditGrade

\begin{Shaded}
\begin{Highlighting}[]
\NormalTok{RawData}\OperatorTok
\StringTok{  }\KeywordTok{filter}\NormalTok{(CreditGrade}\OperatorTok{!=}\StringTok{""}\NormalTok{)}\OperatorTok
\StringTok{  }\KeywordTok{count}\NormalTok{(CreditGrade) }\OperatorTok\StringTok{ }
\StringTok{  }\KeywordTok{ggplot}\NormalTok{(}\KeywordTok{aes}\NormalTok{(}\DataTypeTok{x=}\NormalTok{CreditGrade, }\DataTypeTok{y =}\NormalTok{ n,}\DataTypeTok{fill=}\KeywordTok{sqrt}\NormalTok{(n)))}\OperatorTok{+}
\StringTok{    }\KeywordTok{geom_bar}\NormalTok{(}\DataTypeTok{stat =} \StringTok{"identity"}\NormalTok{)}\OperatorTok{+}
\StringTok{    }\KeywordTok{theme_minimal}\NormalTok{()}\OperatorTok{+}
\StringTok{    }\KeywordTok{theme}\NormalTok{(}\DataTypeTok{axis.text.x =} \KeywordTok{element_text}\NormalTok{( ))}\OperatorTok{+}
\StringTok{    }\KeywordTok{labs}\NormalTok{(}\DataTypeTok{title=}\StringTok{"CreditGrade"}\NormalTok{, }\DataTypeTok{x=}\StringTok{"CreditGrade"}\NormalTok{, }\DataTypeTok{y=}\StringTok{"Count"}\NormalTok{,}\DataTypeTok{fill=}\StringTok{"Sqrt(Count)"}\NormalTok{)}
\end{Highlighting}
\end{Shaded}

\includegraphics{prosperLoanData_files/figure-latex/unnamed-chunk-18-1.pdf}

LP InterestandFees

Pre charge-off cumulative interest and fees paid by the borrower. If the
loan has charged off, this value will exclude any recoveries.

\begin{Shaded}
\begin{Highlighting}[]
\NormalTok{RawData}\OperatorTok
\StringTok{  }\CommentTok{#filter(LP_InterestandFees>quantile(MonthlyLoanPayment,0.001))%>%}
\StringTok{  }\CommentTok{#filter(LP_InterestandFees<quantile(MonthlyLoanPayment,0.999))%>%}
\StringTok{  }\KeywordTok{ggplot}\NormalTok{(}\KeywordTok{aes}\NormalTok{(}\DataTypeTok{x=}\NormalTok{LP_InterestandFees))}\OperatorTok{+}
\StringTok{    }\KeywordTok{geom_histogram}\NormalTok{(}\DataTypeTok{bins=}\DecValTok{300}\NormalTok{,}\DataTypeTok{fill=}\StringTok{"#d53e6c"}\NormalTok{)}\OperatorTok{+}
\StringTok{    }\KeywordTok{theme_minimal}\NormalTok{()}\OperatorTok{+}
\StringTok{    }\KeywordTok{theme}\NormalTok{(}\DataTypeTok{axis.text.x =} \KeywordTok{element_text}\NormalTok{(}\DataTypeTok{size =}\NormalTok{ , }\DataTypeTok{angle=} \DecValTok{45}\NormalTok{, }\DataTypeTok{hjust=}\DecValTok{1}\NormalTok{))}\OperatorTok{+}
\StringTok{    }\KeywordTok{labs}\NormalTok{(}\DataTypeTok{title=}\StringTok{"Pre charge-off cumulative interest and fees paid by the borrower"}\NormalTok{, }\DataTypeTok{x=}\StringTok{"LP InterestandFees"}\NormalTok{, }\DataTypeTok{y=}\StringTok{"Count"}\NormalTok{)}
\end{Highlighting}
\end{Shaded}

\includegraphics{prosperLoanData_files/figure-latex/unnamed-chunk-19-1.pdf}

\begin{Shaded}
\begin{Highlighting}[]
\NormalTok{RawData}\OperatorTok
\StringTok{  }\KeywordTok{filter}\NormalTok{(LP_InterestandFees}\OperatorTok{>}\KeywordTok{quantile}\NormalTok{(MonthlyLoanPayment,}\FloatTok{0.001}\NormalTok{))}\OperatorTok
\StringTok{  }\KeywordTok{filter}\NormalTok{(LP_InterestandFees}\OperatorTok{<}\DecValTok{5000}\NormalTok{)}\OperatorTok
\StringTok{  }\CommentTok{#filter(LP_InterestandFees<quantile(MonthlyLoanPayment,0.999))%>%}
\StringTok{  }\KeywordTok{ggplot}\NormalTok{(}\KeywordTok{aes}\NormalTok{(}\DataTypeTok{x=}\NormalTok{LP_InterestandFees))}\OperatorTok{+}
\StringTok{    }\KeywordTok{geom_histogram}\NormalTok{(}\DataTypeTok{bins=}\DecValTok{300}\NormalTok{,}\DataTypeTok{fill=}\StringTok{"#d53e6c"}\NormalTok{)}\OperatorTok{+}
\StringTok{    }\KeywordTok{theme_minimal}\NormalTok{()}\OperatorTok{+}
\StringTok{    }\KeywordTok{theme}\NormalTok{(}\DataTypeTok{axis.text.x =} \KeywordTok{element_text}\NormalTok{(}\DataTypeTok{size =}\NormalTok{ , }\DataTypeTok{angle=} \DecValTok{45}\NormalTok{, }\DataTypeTok{hjust=}\DecValTok{1}\NormalTok{))}\OperatorTok{+}
\StringTok{    }\KeywordTok{labs}\NormalTok{(}\DataTypeTok{title=}\StringTok{"Pre charge-off cumulative interest and fees paid by the borrower"}\NormalTok{, }\DataTypeTok{x=}\StringTok{"LP InterestandFees"}\NormalTok{, }\DataTypeTok{y=}\StringTok{"Count"}\NormalTok{)}
\end{Highlighting}
\end{Shaded}

\includegraphics{prosperLoanData_files/figure-latex/unnamed-chunk-20-1.pdf}

Investors The number of investors that funded the loan.

\begin{Shaded}
\begin{Highlighting}[]
\NormalTok{Investors=RawData}\OperatorTok{$}\NormalTok{Investors}

\NormalTok{RawData}\OperatorTok
\StringTok{  }\CommentTok{#filter(Investors>quantile(Investors,0.03))%>%}
\StringTok{  }\CommentTok{#filter(Investors<600)%>%}
\StringTok{  }\KeywordTok{filter}\NormalTok{(Investors}\OperatorTok{<}\KeywordTok{quantile}\NormalTok{(Investors,}\FloatTok{0.97}\NormalTok{))}\OperatorTok
\StringTok{  }\KeywordTok{ggplot}\NormalTok{(}\KeywordTok{aes}\NormalTok{(}\DataTypeTok{x=}\NormalTok{Investors))}\OperatorTok{+}
\StringTok{    }\KeywordTok{geom_histogram}\NormalTok{(}\DataTypeTok{bins=}\DecValTok{100}\NormalTok{,}\DataTypeTok{fill=}\StringTok{"#356e9c"}\NormalTok{)}\OperatorTok{+}
\StringTok{    }\KeywordTok{theme_minimal}\NormalTok{()}\OperatorTok{+}
\StringTok{    }\KeywordTok{coord_trans}\NormalTok{(}\DataTypeTok{y=}\StringTok{"sqrt"}\NormalTok{)}\OperatorTok{+}
\StringTok{    }\KeywordTok{theme}\NormalTok{(}\DataTypeTok{axis.text.x =} \KeywordTok{element_text}\NormalTok{(}\DataTypeTok{size =}\NormalTok{ , }\DataTypeTok{angle=} \DecValTok{45}\NormalTok{, }\DataTypeTok{hjust=}\DecValTok{1}\NormalTok{))}\OperatorTok{+}
\StringTok{    }\KeywordTok{labs}\NormalTok{(}\DataTypeTok{title=}\StringTok{"Investors"}\NormalTok{,}
         \DataTypeTok{subtitle=}\StringTok{"The number of investors that funded the loan."}\NormalTok{,}
         \DataTypeTok{x=}\StringTok{"Investors"}\NormalTok{,}
         \DataTypeTok{y=}\StringTok{"Count"}\NormalTok{)}
\end{Highlighting}
\end{Shaded}

\includegraphics{prosperLoanData_files/figure-latex/unnamed-chunk-21-1.pdf}

Employment Status Duration The length in months of the employment status
at the time the listing was created.

\begin{Shaded}
\begin{Highlighting}[]
\NormalTok{EmploymentStatusDuration=RawData}\OperatorTok{$}\NormalTok{EmploymentStatusDuration}
\NormalTok{RawData}\OperatorTok
\StringTok{  }\KeywordTok{filter}\NormalTok{(}\KeywordTok{is.finite}\NormalTok{(EmploymentStatusDuration))}\OperatorTok
\StringTok{  }\KeywordTok{filter}\NormalTok{(EmploymentStatusDuration}\OperatorTok{>}\KeywordTok{quantile}\NormalTok{(EmploymentStatusDuration,}\FloatTok{0.04}\NormalTok{))}\OperatorTok
\StringTok{  }\KeywordTok{filter}\NormalTok{(EmploymentStatusDuration}\OperatorTok{<}\KeywordTok{quantile}\NormalTok{(EmploymentStatusDuration,}\FloatTok{0.96}\NormalTok{))}\OperatorTok
\StringTok{  }\KeywordTok{ggplot}\NormalTok{(}\KeywordTok{aes}\NormalTok{(EmploymentStatusDuration))}\OperatorTok{+}
\StringTok{  }\KeywordTok{geom_histogram}\NormalTok{(}\DataTypeTok{bins=}\DecValTok{156}\NormalTok{,}\DataTypeTok{fill=}\StringTok{"#c54e74"}\NormalTok{)}\OperatorTok{+}
\StringTok{  }\KeywordTok{labs}\NormalTok{(}\DataTypeTok{title=}\StringTok{"Employment Status Duration"}\NormalTok{,}
         \DataTypeTok{subtitle=}\StringTok{"The length in months of the employment status at the time the listing was created."}\NormalTok{,}
         \DataTypeTok{x=}\StringTok{"Employment Status Duration"}\NormalTok{,}
         \DataTypeTok{y=}\StringTok{"Count"}\NormalTok{)}
\end{Highlighting}
\end{Shaded}

\includegraphics{prosperLoanData_files/figure-latex/unnamed-chunk-22-1.pdf}

DebtToIncomeRatio The debt to income ratio of the borrower at the time
the credit profile was pulled. This value is Null if the debt to income
ratio is not available. This value is capped at 10.01 (any debt to
income ratio larger than 1000\% will be returned as 1001\%).

\begin{Shaded}
\begin{Highlighting}[]
\NormalTok{RawData}\OperatorTok
\StringTok{  }\KeywordTok{filter}\NormalTok{(}\KeywordTok{is.finite}\NormalTok{(DebtToIncomeRatio))}\OperatorTok
\StringTok{  }\KeywordTok{filter}\NormalTok{(DebtToIncomeRatio}\OperatorTok{>}\KeywordTok{quantile}\NormalTok{(DebtToIncomeRatio,}\FloatTok{0.01}\NormalTok{))}\OperatorTok
\StringTok{  }\KeywordTok{filter}\NormalTok{(DebtToIncomeRatio}\OperatorTok{<}\KeywordTok{quantile}\NormalTok{(DebtToIncomeRatio,}\FloatTok{0.99}\NormalTok{))}\OperatorTok
\StringTok{  }\KeywordTok{ggplot}\NormalTok{(}\KeywordTok{aes}\NormalTok{(DebtToIncomeRatio))}\OperatorTok{+}
\StringTok{  }\KeywordTok{geom_histogram}\NormalTok{(}\DataTypeTok{binwidth =} \FloatTok{0.01}\NormalTok{,}\DataTypeTok{fill=}\StringTok{"#465498"}\NormalTok{)}
\end{Highlighting}
\end{Shaded}

\includegraphics{prosperLoanData_files/figure-latex/unnamed-chunk-23-1.pdf}

\hypertarget{univariate-analysis}{%
\section{Univariate Analysis}\label{univariate-analysis}}

we can see alot of information from the previos section, some propierts
was expected others was a little superising

examples : It was expected to get some pattern in loan value to tend to
get some pattern (1000 , 5000 , 10000, 15000 ) personally I was expecte
a equal distrupitoin in loan Term, nevertheless it was manily in 36
monthes the in come range for the borrower is normally distrupited as
expected the most interesting thing was the mode in monthly loan payment
around 175 \$ I think there is some requlation in united state reinforce
this amount of monthly payment. something unexpect also the drop in the
entreing date between 2009 and 2010 I think this because of integration
into new system

\#Bivariate Plots Section

Year and Investor per loan
\includegraphics{prosperLoanData_files/figure-latex/Univariate_Plots-1.pdf}

\begin{Shaded}
\begin{Highlighting}[]
\NormalTok{RawData}\OperatorTok
\StringTok{  }\KeywordTok{filter}\NormalTok{(}\OperatorTok{!}\NormalTok{IncomeRange }\OperatorTok\StringTok{ }\KeywordTok{list}\NormalTok{(}\StringTok{"Not displayed"}\NormalTok{,}\StringTok{"$0"}\NormalTok{) )}\OperatorTok
\StringTok{  }\KeywordTok{ggplot}\NormalTok{(}\KeywordTok{aes}\NormalTok{(}\DataTypeTok{x=}\NormalTok{ListingCreationDate,}\DataTypeTok{fill=}\KeywordTok{reorder}\NormalTok{(IncomeRange,}\OperatorTok{-}\KeywordTok{table}\NormalTok{(IncomeRange)[IncomeRange])))}\OperatorTok{+}
\StringTok{  }\KeywordTok{geom_histogram}\NormalTok{(}\DataTypeTok{bins=}\DecValTok{300}\NormalTok{)}\OperatorTok{+}
\StringTok{  }\KeywordTok{coord_trans}\NormalTok{(}\DataTypeTok{y=}\StringTok{"sqrt"}\NormalTok{)}\OperatorTok{+}
\StringTok{  }\KeywordTok{scale_fill_brewer}\NormalTok{(}\DataTypeTok{palette =} \DecValTok{1}\NormalTok{,}\DataTypeTok{direction =} \DecValTok{-1}\NormalTok{)}\OperatorTok{+}
\StringTok{  }\KeywordTok{labs}\NormalTok{(}\DataTypeTok{title=}\StringTok{"Income Range in time "}\NormalTok{,}
       \DataTypeTok{subtitle=}\StringTok{"he income range of the borrower at the time the listing was created."}\NormalTok{,}
       \DataTypeTok{x=}\StringTok{"Listing Creation Date"}\NormalTok{,}
       \DataTypeTok{y=}\StringTok{"Count"}\NormalTok{,}
       \DataTypeTok{fill=}\StringTok{"IncomeRange"}\NormalTok{)}
\end{Highlighting}
\end{Shaded}

\includegraphics{prosperLoanData_files/figure-latex/unnamed-chunk-24-1.pdf}

\begin{Shaded}
\begin{Highlighting}[]
\NormalTok{RawData}\OperatorTok
\StringTok{  }\KeywordTok{filter}\NormalTok{(}\OperatorTok{!}\NormalTok{CreditGrade}\OperatorTok{==}\StringTok{""}\NormalTok{)}\OperatorTok
\StringTok{  }\KeywordTok{ggplot}\NormalTok{(}\KeywordTok{aes}\NormalTok{(}\DataTypeTok{x=}\NormalTok{ListingCreationDate,}\DataTypeTok{fill=}\NormalTok{CreditGrade))}\OperatorTok{+}
\StringTok{  }\KeywordTok{geom_histogram}\NormalTok{(}\DataTypeTok{bins=}\DecValTok{300}\NormalTok{)}\OperatorTok{+}
\StringTok{  }\KeywordTok{coord_trans}\NormalTok{(}\DataTypeTok{y=}\StringTok{"sqrt"}\NormalTok{)}\OperatorTok{+}
\StringTok{  }\KeywordTok{scale_fill_brewer}\NormalTok{(}\DataTypeTok{palette =} \DecValTok{3}\NormalTok{,}\DataTypeTok{type =} \StringTok{"div"}\NormalTok{,}\DataTypeTok{direction =} \DecValTok{-1}\NormalTok{)}\OperatorTok{+}
\StringTok{  }\KeywordTok{labs}\NormalTok{(}\DataTypeTok{title=}\StringTok{"CreditGrade "}\NormalTok{,}
       \DataTypeTok{subtitle=}\StringTok{"The Credit rating that was assigned at the time the listing went live. Applicable for listings pre-2009"}\NormalTok{,}
       \DataTypeTok{x=}\StringTok{"Listing Creation Date"}\NormalTok{,}
       \DataTypeTok{y=}\StringTok{"Count"}\NormalTok{,}
       \DataTypeTok{fill=}\StringTok{"CreditGrade"}\NormalTok{)}
\end{Highlighting}
\end{Shaded}

\includegraphics{prosperLoanData_files/figure-latex/unnamed-chunk-25-1.pdf}

\begin{Shaded}
\begin{Highlighting}[]
\NormalTok{RawData}\OperatorTok
\StringTok{  }\KeywordTok{filter}\NormalTok{(}\OperatorTok{!}\NormalTok{CreditGrade}\OperatorTok{==}\StringTok{""}\NormalTok{)}\OperatorTok
\StringTok{  }\KeywordTok{ggplot}\NormalTok{(}\KeywordTok{aes}\NormalTok{(}\DataTypeTok{x=}\NormalTok{ListingCreationDate,}\DataTypeTok{fill=}\NormalTok{CreditGrade))}\OperatorTok{+}
\StringTok{  }\KeywordTok{geom_density}\NormalTok{(}\DataTypeTok{alpha=}\FloatTok{0.4}\NormalTok{)}\OperatorTok{+}
\StringTok{  }\KeywordTok{scale_fill_brewer}\NormalTok{(}\DataTypeTok{palette =}\StringTok{"Set3"}\NormalTok{ ,}\DataTypeTok{type =} \StringTok{"seq"}\NormalTok{,}\DataTypeTok{direction =} \DecValTok{-1}\NormalTok{)}\OperatorTok{+}
\StringTok{  }\KeywordTok{labs}\NormalTok{(}\DataTypeTok{title=}\StringTok{"CreditGrade "}\NormalTok{,}
       \DataTypeTok{subtitle=}\StringTok{"The Credit rating that was assigned at the time the listing went live. Applicable for listings pre-2009"}\NormalTok{,}
       \DataTypeTok{x=}\StringTok{"Listing Creation Date"}\NormalTok{,}
       \DataTypeTok{y=}\StringTok{"Count"}\NormalTok{,}
       \DataTypeTok{fill=}\StringTok{"CreditGrade"}\NormalTok{)}
\end{Highlighting}
\end{Shaded}

\includegraphics{prosperLoanData_files/figure-latex/unnamed-chunk-26-1.pdf}

Monthly Loan Payment vs time

\begin{Shaded}
\begin{Highlighting}[]
\NormalTok{RawData }\OperatorTok
\StringTok{  }\KeywordTok{filter}\NormalTok{(MonthlyLoanPayment}\OperatorTok{>}\KeywordTok{quantile}\NormalTok{(MonthlyLoanPayment,}\FloatTok{0.02}\NormalTok{))}\OperatorTok
\StringTok{  }\KeywordTok{filter}\NormalTok{(MonthlyLoanPayment}\OperatorTok{<}\KeywordTok{quantile}\NormalTok{(MonthlyLoanPayment,}\FloatTok{0.98}\NormalTok{))}\OperatorTok
\StringTok{  }\KeywordTok{ggplot}\NormalTok{(}\KeywordTok{aes}\NormalTok{(}\DataTypeTok{x=}\NormalTok{ListingCreationDate,}\DataTypeTok{y=}\NormalTok{MonthlyLoanPayment))}\OperatorTok{+}
\StringTok{  }\KeywordTok{geom_jitter}\NormalTok{(}\DataTypeTok{alpha=}\FloatTok{0.01}\NormalTok{,}\DataTypeTok{color=}\StringTok{"red"}\NormalTok{)}\OperatorTok{+}
\StringTok{  }\KeywordTok{geom_smooth}\NormalTok{(}\DataTypeTok{method=}\StringTok{"auto"}\NormalTok{,}\DataTypeTok{level=}\FloatTok{0.5}\NormalTok{,}\DataTypeTok{span=}\FloatTok{0.3}\NormalTok{,}\DataTypeTok{n=}\DecValTok{100}\NormalTok{)}
\end{Highlighting}
\end{Shaded}

\begin{verbatim}
## `geom_smooth()` using method = 'gam' and formula 'y ~ s(x, bs = "cs")'
\end{verbatim}

\includegraphics{prosperLoanData_files/figure-latex/unnamed-chunk-27-1.pdf}

\begin{Shaded}
\begin{Highlighting}[]
\NormalTok{topOcc=}\KeywordTok{levels}\NormalTok{(}\KeywordTok{reorder}\NormalTok{(RawData}\OperatorTok{$}\NormalTok{Occupation, }\OperatorTok{-}\KeywordTok{table}\NormalTok{(RawData}\OperatorTok{$}\NormalTok{Occupation)[RawData}\OperatorTok{$}\NormalTok{Occupation]))[}\DecValTok{0}\OperatorTok{:}\DecValTok{15}\NormalTok{]}
\NormalTok{RawData}\OperatorTok
\StringTok{  }\KeywordTok{filter}\NormalTok{(Occupation }\OperatorTok\StringTok{ }\NormalTok{topOcc)}\OperatorTok
\StringTok{  }\KeywordTok{ggplot}\NormalTok{(}\KeywordTok{aes}\NormalTok{(}\DataTypeTok{x=}\KeywordTok{reorder}\NormalTok{(Occupation, }\OperatorTok{-}\KeywordTok{table}\NormalTok{(Occupation)[Occupation]),}\DataTypeTok{fill=}\NormalTok{LoanStatus))}\OperatorTok{+}
\StringTok{  }\KeywordTok{geom_bar}\NormalTok{()}\OperatorTok{+}
\StringTok{  }\KeywordTok{coord_trans}\NormalTok{(}\DataTypeTok{y=}\StringTok{"sqrt"}\NormalTok{)}\OperatorTok{+}
\StringTok{  }\KeywordTok{theme}\NormalTok{(}\DataTypeTok{axis.text.x =} \KeywordTok{element_text}\NormalTok{(}\DataTypeTok{size=}\DecValTok{6}\NormalTok{, }\DataTypeTok{angle=}\DecValTok{90}\NormalTok{,}\DataTypeTok{hjust =} \DecValTok{1}\NormalTok{))}\OperatorTok{+}
\StringTok{  }\CommentTok{#scale_fill_brewer(palette = "Spectral")+}
\StringTok{  }\KeywordTok{labs}\NormalTok{(}\DataTypeTok{title=}\StringTok{"Occupation "}\NormalTok{, }\DataTypeTok{x=}\StringTok{"Occupation"}\NormalTok{, }\DataTypeTok{y=}\StringTok{"Count"}\NormalTok{,}\DataTypeTok{fill=}\StringTok{" Loan Status"}\NormalTok{)}
\end{Highlighting}
\end{Shaded}

\includegraphics{prosperLoanData_files/figure-latex/unnamed-chunk-28-1.pdf}

levels(RawData\$LoanStatus)\textless{}-c( ``Completed'', ``Cancelled'',
``Chargedoff'', ``Current'', ``Defaulted'', ``FinalPaymentInProgress'',
``Past Due (1-15 days)'', ``Past Due (16-30 days)'', ``Past Due (31-60
days)'', ``Past Due (61-90 days)'', ``Past Due (91-120 days)'', ``Past
Due (\textgreater{}120 days)'' )

levels(RawData\$LoanStatus)

\begin{Shaded}
\begin{Highlighting}[]
\NormalTok{topOcc=}\KeywordTok{levels}\NormalTok{(}\KeywordTok{reorder}\NormalTok{(RawData}\OperatorTok{$}\NormalTok{Occupation,}\OperatorTok{+-}\KeywordTok{table}\NormalTok{(RawData}\OperatorTok{$}\NormalTok{Occupation)[RawData}\OperatorTok{$}\NormalTok{Occupation]))[}\DecValTok{0}\OperatorTok{:}\DecValTok{19}\NormalTok{]}
\NormalTok{RawData }\OperatorTok
\StringTok{  }\KeywordTok{filter}\NormalTok{(LoanStatus}\OperatorTok{!=}\StringTok{"Defaulted"}\NormalTok{)}\OperatorTok
\StringTok{  }\KeywordTok{filter}\NormalTok{(Occupation }\OperatorTok\StringTok{ }\NormalTok{topOcc)}\OperatorTok
\StringTok{  }\CommentTok{#filter(!Occupation=="")%>%}
\StringTok{  }\CommentTok{#filter(!LoanStatus=="")%>%}
\StringTok{  }\KeywordTok{count}\NormalTok{(Occupation,LoanStatus)}\OperatorTok
\StringTok{  }\KeywordTok{ggplot}\NormalTok{(}\KeywordTok{aes}\NormalTok{(}\DataTypeTok{x=}\KeywordTok{reorder}\NormalTok{(Occupation, }\OperatorTok{-}\KeywordTok{table}\NormalTok{(Occupation)[Occupation]),}\DataTypeTok{fill=}\NormalTok{LoanStatus,}\DataTypeTok{y=}\NormalTok{n))}\OperatorTok{+}
\StringTok{    }\KeywordTok{geom_bar}\NormalTok{(}\DataTypeTok{stat =} \StringTok{"identity"}\NormalTok{)}\OperatorTok{+}
\StringTok{    }\KeywordTok{coord_trans}\NormalTok{(}\DataTypeTok{y=}\StringTok{"sqrt"}\NormalTok{)}\OperatorTok{+}
\StringTok{    }\KeywordTok{theme}\NormalTok{(}\DataTypeTok{axis.text.x =} \KeywordTok{element_text}\NormalTok{(}\DataTypeTok{size=}\DecValTok{6}\NormalTok{, }\DataTypeTok{angle=}\DecValTok{90}\NormalTok{,}\DataTypeTok{hjust =} \DecValTok{1}\NormalTok{))}\OperatorTok{+}
\StringTok{    }\KeywordTok{scale_fill_brewer}\NormalTok{(}\DataTypeTok{palette =} \StringTok{"RdYlBu"}\NormalTok{)}\OperatorTok{+}
\StringTok{    }\KeywordTok{labs}\NormalTok{(}\DataTypeTok{title=}\StringTok{"Occupation "}\NormalTok{, }\DataTypeTok{x=}\StringTok{"Occupation"}\NormalTok{, }\DataTypeTok{y=}\StringTok{"Count"}\NormalTok{,}\DataTypeTok{fill=}\StringTok{" Loan Status"}\NormalTok{)}
\end{Highlighting}
\end{Shaded}

\includegraphics{prosperLoanData_files/figure-latex/unnamed-chunk-29-1.pdf}

topOcc=levels(reorder(RawData\(Occupation,+-table(RawData\)Occupation){[}RawData\$Occupation{]})){[}0:19{]}
RawData \%\textgreater{}\%
filter(LoanStatus!=``Defaulted'')\%\textgreater{}\% filter(Occupation
\%in\% topOcc)\%\textgreater{}\%
\#filter(!Occupation=="``)\%\textgreater{}\%
\#filter(!LoanStatus==''``)\%\textgreater{}\%
count(Occupation,LoanStatus)\%\textgreater{}\%
ggplot(aes(x=reorder(Occupation,
-table(Occupation){[}Occupation{]}),fill=LoanStatus,y=n))+
geom\_bar(stat =''identity``)+ coord\_trans(y=''sqrt``)+
theme(axis.text.x = element\_text(size=6, angle=90,hjust = 1))+
scale\_fill\_brewer(palette =''RdYlBu``)+ labs(title=''Occupation ``,
x=''Occupation``, y=''Count``,fill='' Loan Status")

Term vs Time Creating,

Term vs Investors

\begin{Shaded}
\begin{Highlighting}[]
\NormalTok{RawData}\OperatorTok
\StringTok{  }
\StringTok{  }\KeywordTok{ggplot}\NormalTok{(}\KeywordTok{aes}\NormalTok{(}\DataTypeTok{x=}\NormalTok{LoanOriginalAmount, }\DataTypeTok{y=}\KeywordTok{as.factor}\NormalTok{(Term), }\DataTypeTok{fill=}\FloatTok{0.5} \OperatorTok{-}\StringTok{ }\KeywordTok{abs}\NormalTok{(}\FloatTok{0.5}\OperatorTok{-}\NormalTok{..ecdf..))) }\OperatorTok{+}
\StringTok{    }\KeywordTok{stat_density_ridges}\NormalTok{(}\DataTypeTok{geom =} \StringTok{"density_ridges_gradient"}\NormalTok{, }\DataTypeTok{calc_ecdf =} \OtherTok{TRUE}\NormalTok{) }\OperatorTok{+}
\StringTok{    }\KeywordTok{scale_fill_viridis}\NormalTok{(}\DataTypeTok{name =} \StringTok{"Tail probability"}\NormalTok{, }\DataTypeTok{direction =} \DecValTok{-1}\NormalTok{)}\OperatorTok{+}
\StringTok{    }\KeywordTok{theme}\NormalTok{(}\DataTypeTok{axis.text.x =} \KeywordTok{element_text}\NormalTok{(}\DataTypeTok{size =}\NormalTok{ , }\DataTypeTok{angle=} \DecValTok{45}\NormalTok{, }\DataTypeTok{hjust=}\DecValTok{1}\NormalTok{))}\OperatorTok{+}
\StringTok{    }\KeywordTok{labs}\NormalTok{(}\DataTypeTok{title=}\StringTok{"Investors"}\NormalTok{,}
         \DataTypeTok{subtitle=}\StringTok{"The number of investors that funded the loan."}\NormalTok{,}
         \DataTypeTok{x=}\StringTok{"Investors"}\NormalTok{,}
         \DataTypeTok{y=}\StringTok{"Term"}\NormalTok{)}
\end{Highlighting}
\end{Shaded}

\begin{verbatim}
## Picking joint bandwidth of 522
\end{verbatim}

\includegraphics{prosperLoanData_files/figure-latex/unnamed-chunk-30-1.pdf}
DebtToIncomeRatio vs BorrowerRate

RawData\%\textgreater{}\%
filter(BorrowerRate\textgreater{}quantile(BorrowerRate,0.01))\%\textgreater{}\%
filter(BorrowerRate\textless{}quantile(BorrowerRate,0.99))\%\textgreater{}\%
\#count(BorrowerRate)\%\textgreater{}\% ggplot(aes(x=BorrowerRate))+
geom\_histogram(bins=40,fill=``\#563ec3'')+ theme\_minimal()+
theme(axis.text.x = element\_text(size = , angle= 45, hjust=1))+
labs(title=``The Borrower's interest rate for this loan.'', x=``Borrower
Rate'', y=``Count'')

\begin{Shaded}
\begin{Highlighting}[]
\NormalTok{RawData}\OperatorTok
\StringTok{  }\KeywordTok{filter}\NormalTok{(}\KeywordTok{is.finite}\NormalTok{(DebtToIncomeRatio))}\OperatorTok
\StringTok{  }\KeywordTok{filter}\NormalTok{(DebtToIncomeRatio}\OperatorTok{>}\KeywordTok{quantile}\NormalTok{(DebtToIncomeRatio,}\FloatTok{0.01}\NormalTok{))}\OperatorTok
\StringTok{  }\KeywordTok{filter}\NormalTok{(DebtToIncomeRatio}\OperatorTok{<}\KeywordTok{quantile}\NormalTok{(DebtToIncomeRatio,}\FloatTok{0.99}\NormalTok{))}\OperatorTok
\StringTok{  }\KeywordTok{filter}\NormalTok{(BorrowerRate}\OperatorTok{>}\KeywordTok{quantile}\NormalTok{(BorrowerRate,}\FloatTok{0.01}\NormalTok{))}\OperatorTok
\StringTok{  }\KeywordTok{filter}\NormalTok{(BorrowerRate}\OperatorTok{<}\KeywordTok{quantile}\NormalTok{(BorrowerRate,}\FloatTok{0.99}\NormalTok{))}\OperatorTok
\StringTok{  }\KeywordTok{ggplot}\NormalTok{(}\KeywordTok{aes}\NormalTok{(}\DataTypeTok{x=}\NormalTok{DebtToIncomeRatio,}\DataTypeTok{y=}\NormalTok{BorrowerRate))}\OperatorTok{+}
\StringTok{    }\KeywordTok{geom_jitter}\NormalTok{(}\DataTypeTok{alpha=}\FloatTok{0.05}\NormalTok{,}\DataTypeTok{color=}\StringTok{"#f65436"}\NormalTok{)}\OperatorTok{+}
\StringTok{    }\KeywordTok{geom_smooth}\NormalTok{()}
\end{Highlighting}
\end{Shaded}

\begin{verbatim}
## `geom_smooth()` using method = 'gam' and formula 'y ~ s(x, bs = "cs")'
\end{verbatim}

\includegraphics{prosperLoanData_files/figure-latex/unnamed-chunk-31-1.pdf}

\begin{Shaded}
\begin{Highlighting}[]
\KeywordTok{cor.test}\NormalTok{(RawData}\OperatorTok{$}\NormalTok{DebtToIncomeRatio,RawData}\OperatorTok{$}\NormalTok{BorrowerRate)}
\end{Highlighting}
\end{Shaded}

\begin{verbatim}
## 
##  Pearson's product-moment correlation
## 
## data:  RawData$DebtToIncomeRatio and RawData$BorrowerRate
## t = 20.465, df = 105380, p-value < 2.2e-16
## alternative hypothesis: true correlation is not equal to 0
## 95 percent confidence interval:
##  0.05690080 0.06892819
## sample estimates:
##        cor 
## 0.06291678
\end{verbatim}

\begin{Shaded}
\begin{Highlighting}[]
\NormalTok{RawData}\OperatorTok
\StringTok{  }\KeywordTok{filter}\NormalTok{(}\KeywordTok{is.finite}\NormalTok{(LoanOriginalAmount))}\OperatorTok
\StringTok{  }\KeywordTok{filter}\NormalTok{(LoanOriginalAmount}\OperatorTok{>}\KeywordTok{quantile}\NormalTok{(LoanOriginalAmount,}\FloatTok{0.01}\NormalTok{))}\OperatorTok
\StringTok{  }\KeywordTok{filter}\NormalTok{(LoanOriginalAmount}\OperatorTok{<}\KeywordTok{quantile}\NormalTok{(LoanOriginalAmount,}\FloatTok{0.99}\NormalTok{))}\OperatorTok
\StringTok{  }\KeywordTok{filter}\NormalTok{(BorrowerRate}\OperatorTok{>}\KeywordTok{quantile}\NormalTok{(BorrowerRate,}\FloatTok{0.01}\NormalTok{))}\OperatorTok
\StringTok{  }\KeywordTok{filter}\NormalTok{(BorrowerRate}\OperatorTok{<}\KeywordTok{quantile}\NormalTok{(BorrowerRate,}\FloatTok{0.99}\NormalTok{))}\OperatorTok
\StringTok{  }\KeywordTok{ggplot}\NormalTok{(}\KeywordTok{aes}\NormalTok{(}\DataTypeTok{x=}\NormalTok{LoanOriginalAmount,}\DataTypeTok{y=}\NormalTok{BorrowerRate))}\OperatorTok{+}
\StringTok{    }\KeywordTok{geom_point}\NormalTok{(}\DataTypeTok{alpha=}\FloatTok{0.05}\NormalTok{,}\DataTypeTok{color=}\StringTok{"#f65436"}\NormalTok{)}\OperatorTok{+}
\StringTok{    }\KeywordTok{geom_smooth}\NormalTok{()}
\end{Highlighting}
\end{Shaded}

\begin{verbatim}
## `geom_smooth()` using method = 'gam' and formula 'y ~ s(x, bs = "cs")'
\end{verbatim}

\includegraphics{prosperLoanData_files/figure-latex/unnamed-chunk-33-1.pdf}

\begin{Shaded}
\begin{Highlighting}[]
\KeywordTok{cor.test}\NormalTok{(RawData}\OperatorTok{$}\NormalTok{LoanOriginalAmount,RawData}\OperatorTok{$}\NormalTok{BorrowerRate)}
\end{Highlighting}
\end{Shaded}

\begin{verbatim}
## 
##  Pearson's product-moment correlation
## 
## data:  RawData$LoanOriginalAmount and RawData$BorrowerRate
## t = -117.58, df = 113940, p-value < 2.2e-16
## alternative hypothesis: true correlation is not equal to 0
## 95 percent confidence interval:
##  -0.3341283 -0.3237719
## sample estimates:
##        cor 
## -0.3289599
\end{verbatim}

\#Bivariate Analysis

what interesing her that in creditGrade there was type NC only before
2009

\hypertarget{multivariate-plots-section}{%
\section{Multivariate Plots Section}\label{multivariate-plots-section}}

ocppuaptin

\begin{Shaded}
\begin{Highlighting}[]
\NormalTok{RawData}\OperatorTok
\StringTok{  }\KeywordTok{filter}\NormalTok{(Occupation }\OperatorTok\StringTok{ }\NormalTok{topOcc)}\OperatorTok
\StringTok{  }\KeywordTok{ggplot}\NormalTok{(}\KeywordTok{aes}\NormalTok{(}\DataTypeTok{x=}\KeywordTok{reorder}\NormalTok{(Occupation,}\OperatorTok{-}\KeywordTok{table}\NormalTok{(Occupation)[Occupation]),}
             \DataTypeTok{y=}\NormalTok{LoanOriginalAmount,}
             \DataTypeTok{fill=}\KeywordTok{reorder}\NormalTok{(EmploymentStatus,}\OperatorTok{-}\KeywordTok{table}\NormalTok{(EmploymentStatus)[EmploymentStatus])}
\NormalTok{             ))}\OperatorTok{+}
\StringTok{    }\CommentTok{#geom_bar(stat="identity",alpha=0.5)+}
\StringTok{    }\KeywordTok{geom_bar}\NormalTok{(}\DataTypeTok{stat=}\StringTok{"identity"}\NormalTok{)}\OperatorTok{+}
\StringTok{    }\KeywordTok{scale_fill_viridis_d}\NormalTok{(}\DataTypeTok{option =} \StringTok{"D"}\NormalTok{,}\DataTypeTok{direction =} \DecValTok{-1}\NormalTok{)}\OperatorTok{+}
\StringTok{    }\CommentTok{#coord_trans(y="sqrt")+}
\StringTok{    }\KeywordTok{theme}\NormalTok{(}\DataTypeTok{axis.text.x =} \KeywordTok{element_text}\NormalTok{(}\DataTypeTok{size=}\DecValTok{6}\NormalTok{, }\DataTypeTok{angle=}\DecValTok{90}\NormalTok{,}\DataTypeTok{hjust =} \DecValTok{1}\NormalTok{))}\OperatorTok{+}
\StringTok{    }\KeywordTok{labs}\NormalTok{(}\DataTypeTok{title=}\StringTok{"Occupation, Employment status, and total orignal ammont"}\NormalTok{, }
         \DataTypeTok{x=}\StringTok{"Occupation"}\NormalTok{,}
         \DataTypeTok{y=}\StringTok{"Total Original Amount"}\NormalTok{,}
         \DataTypeTok{fill=}\StringTok{" EmploymentStatus"}\NormalTok{)}
\end{Highlighting}
\end{Shaded}

\includegraphics{prosperLoanData_files/figure-latex/unnamed-chunk-35-1.pdf}

\begin{Shaded}
\begin{Highlighting}[]
\NormalTok{RawData}\OperatorTok
\StringTok{  }\KeywordTok{filter}\NormalTok{(ProsperRating..Alpha.}\OperatorTok{!=}\StringTok{""}\NormalTok{)}\OperatorTok
\StringTok{  }\KeywordTok{filter}\NormalTok{(}\KeywordTok{is.finite}\NormalTok{(MonthlyLoanPayment))}\OperatorTok
\StringTok{  }\KeywordTok{filter}\NormalTok{(BorrowerRate}\OperatorTok{>}\KeywordTok{quantile}\NormalTok{(BorrowerRate,}\FloatTok{0.01}\NormalTok{))}\OperatorTok
\StringTok{  }\KeywordTok{filter}\NormalTok{(BorrowerRate}\OperatorTok{<}\KeywordTok{quantile}\NormalTok{(BorrowerRate,}\FloatTok{0.99}\NormalTok{))}\OperatorTok
\StringTok{  }\KeywordTok{filter}\NormalTok{(MonthlyLoanPayment}\OperatorTok{>}\KeywordTok{quantile}\NormalTok{(MonthlyLoanPayment,}\FloatTok{0.01}\NormalTok{))}\OperatorTok
\StringTok{  }\KeywordTok{filter}\NormalTok{(MonthlyLoanPayment}\OperatorTok{<}\KeywordTok{quantile}\NormalTok{(MonthlyLoanPayment,}\FloatTok{0.99}\NormalTok{))}\OperatorTok
\StringTok{  }\CommentTok{#filter(Investors>quantile(Investors,0.01))%>%}
\StringTok{  }\CommentTok{#filter(Investors<quantile(Investors,0.99))%>%}
\StringTok{  }\CommentTok{#filter(LoanOriginalAmount!=0)%>%}
\StringTok{  }\KeywordTok{ggplot}\NormalTok{(}\KeywordTok{aes}\NormalTok{(}\DataTypeTok{y=}\NormalTok{MonthlyLoanPayment   ,}\DataTypeTok{x=}\NormalTok{BorrowerRate  ,}\DataTypeTok{color=}\NormalTok{ProsperRating..Alpha.  ))}\OperatorTok{+}
\StringTok{      }\KeywordTok{geom_point}\NormalTok{(}\DataTypeTok{alpha=}\FloatTok{0.2}\NormalTok{)}\OperatorTok{+}
\StringTok{      }\KeywordTok{scale_color_viridis_d}\NormalTok{(}\DataTypeTok{option =} \StringTok{"D"}\NormalTok{)}\OperatorTok{+}
\StringTok{      }\CommentTok{#coord_trans(y="sqrt")+}
\StringTok{      }\KeywordTok{theme}\NormalTok{(}\DataTypeTok{axis.text.x =} \KeywordTok{element_text}\NormalTok{(}\DataTypeTok{size=}\DecValTok{6}\NormalTok{, }\DataTypeTok{angle=}\FloatTok{0.1}\NormalTok{,}\DataTypeTok{hjust =} \DecValTok{1}\NormalTok{))}\OperatorTok{+}
\StringTok{      }\KeywordTok{labs}\NormalTok{(}\DataTypeTok{title=}\StringTok{"BorrowerRate,Monthly Loan Payment, Prosper Rating"}\NormalTok{, }
           \DataTypeTok{x=}\StringTok{"Borrower Rate"}\NormalTok{,}
           \DataTypeTok{color=}\StringTok{"Prosper Rating "}\NormalTok{,}
           \DataTypeTok{y=}\StringTok{"Monthly Loan Payment"}\NormalTok{,}
           \DataTypeTok{alpha=}\DecValTok{1}\NormalTok{)}
\end{Highlighting}
\end{Shaded}

\includegraphics{prosperLoanData_files/figure-latex/unnamed-chunk-36-1.pdf}

\hypertarget{multivariate-analysis}{%
\section{Multivariate Analysis}\label{multivariate-analysis}}

we can see the pattern in between montly loan payment , borrower rate
and the proposer rate.

\begin{center}\rule{0.5\linewidth}{\linethickness}\end{center}

\hypertarget{final-plots-and-summary.}{%
\section{Final Plots and Summary.}\label{final-plots-and-summary.}}

\hypertarget{plot-one}{%
\subsubsection{Plot One}\label{plot-one}}

\begin{Shaded}
\begin{Highlighting}[]
\NormalTok{RawData }\OperatorTok
\StringTok{  }\KeywordTok{filter}\NormalTok{(Occupation}\OperatorTok{!=}\StringTok{"Other"}\NormalTok{ )}\OperatorTok
\StringTok{  }\KeywordTok{filter}\NormalTok{(Occupation}\OperatorTok{!=}\StringTok{"Professional"}\NormalTok{ )}\OperatorTok
\StringTok{  }\KeywordTok{filter}\NormalTok{(Occupation}\OperatorTok{!=}\StringTok{""}\NormalTok{)}\OperatorTok
\StringTok{  }\KeywordTok{count}\NormalTok{(Occupation) }\OperatorTok\StringTok{ }
\StringTok{  }\KeywordTok{mutate}\NormalTok{(}\DataTypeTok{Occupation=}\KeywordTok{reorder}\NormalTok{(Occupation,}\OperatorTok{-}\NormalTok{n))}\OperatorTok\StringTok{ }
\StringTok{  }\KeywordTok{filter}\NormalTok{(n}\OperatorTok{>}\KeywordTok{quantile}\NormalTok{(n,}\FloatTok{0.70}\NormalTok{))}\OperatorTok
\StringTok{  }\KeywordTok{ggplot}\NormalTok{(}\KeywordTok{aes}\NormalTok{(}\DataTypeTok{x=}\NormalTok{Occupation, }\DataTypeTok{y =}\NormalTok{ n,}\DataTypeTok{fill=}\KeywordTok{sqrt}\NormalTok{(n)))}\OperatorTok{+}
\StringTok{    }\KeywordTok{geom_bar}\NormalTok{(}\DataTypeTok{stat =} \StringTok{"identity"}\NormalTok{)}\OperatorTok{+}
\StringTok{    }\CommentTok{#scale_fill_viridis(option = "D")+}
\StringTok{    }\KeywordTok{theme_minimal}\NormalTok{()}\OperatorTok{+}
\StringTok{    }\KeywordTok{theme}\NormalTok{(}\DataTypeTok{axis.text.x =} \KeywordTok{element_text}\NormalTok{(}\DataTypeTok{size =} \DecValTok{8}\NormalTok{, }\DataTypeTok{angle=} \DecValTok{60}\NormalTok{, }\DataTypeTok{hjust=}\DecValTok{1}\NormalTok{))}\OperatorTok{+}
\StringTok{    }\KeywordTok{labs}\NormalTok{(}\DataTypeTok{title=}\StringTok{"Occupation "}\NormalTok{, }\DataTypeTok{x=}\StringTok{"Occupation"}\NormalTok{, }\DataTypeTok{y=}\StringTok{"Count"}\NormalTok{,}\DataTypeTok{fill=}\StringTok{"Sqrt(Count)"}\NormalTok{)}
\end{Highlighting}
\end{Shaded}

\includegraphics{prosperLoanData_files/figure-latex/unnamed-chunk-37-1.pdf}

this plot describe what is the distruption of loans count regarding the
profession after discared the misinformative occpuation like (other,
profession) so we can suggest wich field of worker in the next campigns
to target to get some more sales

\hypertarget{plot-two}{%
\subsubsection{Plot Two}\label{plot-two}}

\begin{Shaded}
\begin{Highlighting}[]
\NormalTok{RawData}\OperatorTok
\StringTok{  }\KeywordTok{ggplot}\NormalTok{(}\KeywordTok{aes}\NormalTok{(}\DataTypeTok{x=}\NormalTok{ListingCreationDate,}\DataTypeTok{fill=}\KeywordTok{as.factor}\NormalTok{(Term)))}\OperatorTok{+}
\StringTok{  }\KeywordTok{geom_density}\NormalTok{(}\DataTypeTok{alpha=}\FloatTok{0.6}\NormalTok{)}\OperatorTok{+}
\StringTok{  }\KeywordTok{scale_fill_brewer}\NormalTok{(}\DataTypeTok{palette =} \StringTok{"Set2"}\NormalTok{,}\DataTypeTok{direction =} \DecValTok{1}\NormalTok{)}\OperatorTok{+}
\StringTok{  }\KeywordTok{labs}\NormalTok{(}\DataTypeTok{title=}\StringTok{"Term "}\NormalTok{,}
       \DataTypeTok{subtitle=}\StringTok{"The length of the loan expressed in months."}\NormalTok{,}
       \DataTypeTok{x=}\StringTok{"Listing Creation Date"}\NormalTok{,}
       \DataTypeTok{y=}\StringTok{"Density"}\NormalTok{,}
       \DataTypeTok{fill=}\StringTok{"CreditGrade"}\NormalTok{)}
\end{Highlighting}
\end{Shaded}

\includegraphics{prosperLoanData_files/figure-latex/unnamed-chunk-38-1.pdf}

In this plot we can answer our quesiotn in univaraible examination which
was a little bit surprising why the 36 is the most likable loan we can
see that other types are only started after 2009 ( maybe of reugaltion)
also the other type of loan ( 12 ) is no available any more inspite of
it was very popular I think there's a good markt opprtuinity in prompt
again this type of Loan

\hypertarget{plot-loanoriginalamount-vs-monthlyloanpayment-term}{%
\subsubsection{Plot LoanOriginalAmount vs MonthlyLoanPayment ,
Term}\label{plot-loanoriginalamount-vs-monthlyloanpayment-term}}

\begin{Shaded}
\begin{Highlighting}[]
\NormalTok{RawData}\OperatorTok
\StringTok{  }\KeywordTok{filter}\NormalTok{(MonthlyLoanPayment}\OperatorTok{>}\KeywordTok{quantile}\NormalTok{(MonthlyLoanPayment,}\FloatTok{0.01}\NormalTok{))}\OperatorTok
\StringTok{  }\KeywordTok{filter}\NormalTok{(MonthlyLoanPayment}\OperatorTok{<}\KeywordTok{quantile}\NormalTok{(MonthlyLoanPayment,}\FloatTok{0.99}\NormalTok{))}\OperatorTok
\StringTok{  }\KeywordTok{filter}\NormalTok{(LoanOriginalAmount}\OperatorTok{!=}\DecValTok{0}\NormalTok{)}\OperatorTok
\StringTok{  }\KeywordTok{ggplot}\NormalTok{(}\KeywordTok{aes}\NormalTok{(}\DataTypeTok{y=}\NormalTok{LoanOriginalAmount,}\DataTypeTok{x=}\NormalTok{MonthlyLoanPayment,}\DataTypeTok{color=}\KeywordTok{as.factor}\NormalTok{(Term)))}\OperatorTok{+}
\StringTok{      }\KeywordTok{geom_point}\NormalTok{(}\DataTypeTok{alpha=}\FloatTok{0.3}\NormalTok{)}\OperatorTok{+}
\StringTok{      }\KeywordTok{scale_color_viridis_d}\NormalTok{(}\DataTypeTok{option=}\StringTok{"D"}\NormalTok{)}\OperatorTok{+}
\StringTok{      }\KeywordTok{scale_fill_viridis_d}\NormalTok{(}\DataTypeTok{option =} \StringTok{"D"}\NormalTok{,}\DataTypeTok{direction =} \DecValTok{-1}\NormalTok{)}\OperatorTok{+}
\StringTok{      }\CommentTok{#coord_trans(y="sqrt")+}
\StringTok{      }\KeywordTok{theme}\NormalTok{(}\DataTypeTok{axis.text.x =} \KeywordTok{element_text}\NormalTok{(}\DataTypeTok{size=}\DecValTok{6}\NormalTok{, }\DataTypeTok{angle=}\DecValTok{90}\NormalTok{,}\DataTypeTok{hjust =} \DecValTok{1}\NormalTok{))}\OperatorTok{+}
\StringTok{      }\KeywordTok{labs}\NormalTok{(}\DataTypeTok{title=}\StringTok{"Occupation, Employment status, and loan orignal ammont"}\NormalTok{, }
           \DataTypeTok{x=}\StringTok{"Monthly Loan Payment"}\NormalTok{,}
           \DataTypeTok{y=}\StringTok{"Loan Original Amount"}\NormalTok{,}
           \DataTypeTok{color=}\StringTok{"Term in Monthes"}\NormalTok{)}
\end{Highlighting}
\end{Shaded}

\includegraphics{prosperLoanData_files/figure-latex/unnamed-chunk-39-1.pdf}

In the preivous plot we can easly see the relation between the three
variabls and the lineartiy is obvious

\begin{center}\rule{0.5\linewidth}{\linethickness}\end{center}

\hypertarget{reflection}{%
\section{Reflection}\label{reflection}}

we have explored myn variabls and this data set is really very
informative, I wish i had the time to dig more and explore an extra
ready variblas or calculated supported ones espeically the last plot
descripe the realtion between the variables


\end{document}
